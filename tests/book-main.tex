\documentclass[output=book
  ,nonflat
  ,modfonts,
  ,colorlinks
  ,undecapitalize 
  ,collection
  ,showindex
  ,draftmode
  ,openreview
  ,nobabel
  ,booklanguage=french
  ]{langsci/langscibook}

\usepackage{./langsci/styles/langsci-basic}
\usepackage{./langsci/styles/langsci-tbls}
\usepackage{./langsci/styles/langsci-forest-setup}
\usepackage{./langsci/styles/langsci-glyphs} 
\usepackage{./langsci/styles/langsci-linguex} 
\usepackage{./langsci/styles/langsci-gb4e}
\usepackage{./langsci/styles/jambox} 
\usepackage{./langsci/styles/langsci-optional} 
\usepackage{./langsci/styles/langsci-tobi} 
\usepackage{./langsci/styles/langsci-lgr} 

\usepackage{lipsum}
\usepackage{multicol}
\usepackage{minibox}
\usepackage{forest}
\usepackage{tikz}
\usepackage{pgfplots}

\usepackage{./langsci/styles/avm}
\avmoptions{center} 
\avmfont{\scshape}
\avmvalfont{\normalfont}
\avmsortfont{\normalfont\itshape}

\renewcommand{\lsISBNhardcover}{999-3-123456-99-9}
\renewcommand{\lsISBNsoftcover}{999-3-123456-99-9}
\renewcommand{\lsISBNdigital}{978-3-946234-65-4}
 

\bibliography{localbibliography}
 

\BackBody{
What causes a language to be the way it is? Some features are universal, some are inherited, others are borrowed, and yet others are internally innovated. But no matter where a bit of language is from, it will only exist if it has been diffused and kept in circulation through social interaction in the history of a community. This book makes the case that a proper understanding of the ontology of language systems has to be grounded in the causal mechanisms by which linguistic items are socially transmitted, in communicative contexts. A biased transmission model provides a basis for understanding why certain things and not others are likely to develop, spread, and stick in languages. Because bits of language are always parts of systems, we also need to show how it is that items of knowledge and behavior become structured wholes. The book argues that to achieve this, we need to see how causal processes apply in multiple frames or 'time scales' simultaneously, and we need to understand and address each and all of these frames in our work on language. This forces us to confront implications that are not always comfortable: for example, that "a language" is not a real thing but a convenient fiction, that language-internal and language-external processes have a lot in common, and that tree diagrams are poor conceptual tools for understanding the history of languages. By exploring avenues for clear solutions to these problems, this book suggests a conceptual framework for ultimately explaining, in causal terms, what languages are like and why they are like that.
}

\dedication{Für Alma, Ariel, Block, Frau Brüggenolte, Chopin, Christina, Doro, Edgar, Elena, Elin, Emma, den ehemaligen FCR Duisburg, Frida, Gabriele, Hamlet, Helmut Schmidt, Henry, Ian Kilmister, Ingeborg, Ischariot, Jean-Pierre, Johan, Kurt, Lemmy, Liv, Marina, Martin, Mats, Mausi, Michelle, Nadezhda, Herrn Oelschlägel, Oma, Opa, Pavel, Philly, Sarah, Scully, Stig, Tania, Tante Klärchen, Tarek, Tatjana, Herrn Uhl, Ullis schreckhaften Hund, Vanessa und so. Wenn das schonmal klar sein würde.}

 
\title{Test for the langsci-* packages}
\author{Lang Uage\and Science\lastand Press}
\renewcommand{\lsYear}{1999} 
  


\begin{document}
\maketitle 
\tableofcontents
\mainmatter

\part{Part test}
\chapter{Tests} 
\section{langsci-gb4e}
% \subsection{Environments and syntactic sugar}
% \begin{exe}
\ex this is a beginexe example
\end{exe}

\begin{exe}
\ex
\gll  this is a beginexe example which goes to the end of the line but not beyond\\
     This Is A Beginexe Example Which Goes To The End Of The Line But Not Beyond\\
\glt  `this is a beginexe example which goes to the end of the line but not beyond'
\end{exe}

 
\ea 
\label{ex:13-227} 
\gll anú eesó míiš ki \textbf{(nu)} \textbf{dhoóṛ} \textbf{yhóol-u} \textbf{de}\\
\textsc{3msg.prox.nom} \textsc{rem.msg.nom} man \textsc{comp} \textsc{3msg.prox.nom} yesterday come.\textsc{pfv"=msg} \textsc{pst}\\
\glt `This line must not flow into the margin.'
\z
 

\begin{exe} 
\ex
\gll anú eesó míiš ki \textbf{(nu)} \textbf{dhoóṛ} \textbf{yhóol-u} \textbf{de}\\
\textsc{3msg.prox.nom} \textsc{rem.msg.nom} man \textsc{comp} \textsc{3msg.prox.nom} yesterday come.\textsc{pfv"=msg} \textsc{pst}\\
\glt `This line must not flow into the margin either.'
\end{exe}



% % % \input{gb4e-tests/exbegin}
% \input{gb4e-tests/subex}
% \input{gb4e-tests/xlist}
% \input{gb4e-tests/subsubex}
% \input{gb4e-tests/exi}
% \input{gb4e-tests/ea}
% \input{gb4e-tests/judgment}
% % align table
% align parbox
% align tree
% align for subex
 

\eabox{
\begin{tabular}{ll}
	this \\
	is \\
	vertically\\
	large\\
	content
      \end{tabular}
}

\eabox[0pt]{
\begin{tabular}{ll}
	this is content \\
	with a different \\
	vertical\\
	offset
      \end{tabular}
}


\eabox{
\fbox{\parbox{.6\textwidth}{a\\[4mm]a \fbox{b c \fbox{d e} f} g \fbox{h i \fbox j} k\vspace*{5mm}}}
}

% 
% \eabox[-.85\baselineskip]{
% \begin{forest} for tree={grow=north}
% [, phantom  [  , phantom [, phantom  [{\hspace{5mm} = [a.ˈpa.pa]}]]] [σ [V [a]] [C [p]] ] [σ [V [a]] [C]]  [, phantom [\LARGE{+}]] [σ [,phantom  [a]] [C [p]] [V [a]][C] [,phantom]]]
% \end{forest} 
% }

\newpage
\ea
\begin{xlist}
\exbox[-.7\baselineskip]{\begin{tabular}{ll}
	this \\
	is \\
	a sub example\\
      \end{tabular}}
\exbox[-.66\baselineskip]{\begin{tabular}{ll}
	this \\
	is \\
	a\\
	sub\\
	example
      \end{tabular}}
\end{xlist}
\z

% \subsection{linguex}

\Lsciex. abc
\a.  abc
\b. def
  \a. ghi
  \b. jkl
  \b. mno
  

\Lsciex. 
\gll abc def\\
ghi jkl\\
\glt `no sense attached'

 
\Lsciex. 
\ag. fgh ijk\\
     asd fgh\\
\bg. jkl uio\\
     qwe ert\\
     
     
     
\ea this is gb4e
 \ea indented\\
 	\gll abc def\\
         ghi jkl\\
  \z
\z  
     
% 
%  
% \subsection{Glosses}
% \input{gb4e-tests/gll}
% \input{gb4e-tests/glll}
% \ea
 \gllll dies ist ein gllll Test\\
      this is a gllll Test\\
      \textsc{det} \textsc{cop} \textsc{det} \textsc{n} \textsc{n}\\
     a b c d e\\ 
 \glt `This is a test for quadruple aligned lines (gllll)'
\z
% \ea
\gll Ein potentieller Mega-Deal in der Industrie sorgt an der B{\"o}rse f{\"u}r Euphorie.\\
        \INDF.\textsc{m}.\SG.\NOM{} potential.\textsc{m}.\SG.\NOM{} mega-deal in the industry care-3\SG.\PRS.\IND{} \LOC{} \DEF.\F.\SG.\DAT{} stock.exchange for euphoria\\
\glt `A potential Mega-Deal in industry causes euphoria at the stock exchange' (Der Spiegel, 2014-04-24) (Very long lines should break nicely)
\z

% \input{gb4e-tests/exewidth}
%  
% \subsection{Footnotes}
% \input{gb4e-tests/footnote}
% \input{gb4e-tests/footnotemark}
% Similar things can be found in German\footnote{%
\eafirst
    \gll Es regnet\\
    it rains\\
    \glt `It rains.'
  \zlast
} and in Spanish\footnote{%
\eafirst
    \ea
      \gll llueve\\
      rains\\
      \glt it rains
    \ex
      \gll nieve\\
      snows\\
      \glt it snows
    \z
  \z
}
 
% \subsection{Metadata}
% \input{gb4e-tests/languageinfo}
% \subsection{Boxing}
% \input{gb4e-tests/xbox} 
% \subsection{Jambox}
% %%%%%%%%%%%%%%%%%% JAMBOX: RIGHT-COLUMN ANNOTATIONS %%%%%%%%%%%%%%%%
%
% Alexis Dimitriadis
%
% This is version 0.3 (informal release, Nov. 2003).
%
% Line up material a fixed distance from the right margin.  For annotating
% example sentences, usually with a short note in parentheses.
% May overflow to the left or right, or line up on the next line as necessary.
%
% \jambox[width]{text}	Align 'text' starting 'width' distance from the
%			right margin (default \the\jamwidth).
% \jam(something)	Align a note delimited by parentheses (which are
%			retained).  No optional argument.
% \jambox*{text}        Set \jamwidth to the width of 'text', then align it.
%			(\jamwidth stays set for the rest of the environment).
%
% Notes:
%
% Distance from the right margin can be set to an explicit amount, or to the
% width of some piece of text, as follows:
%
% \jamwidth=2in\relax      Or
% \settowidth\jamwidth {(``annotation'')}
%
% \jamwidth is locally scoped, so it can be set globally or inside an example
% environment.
%
% BUG: Not compatible with ragged-right mode.
%
% Incompatibilities: Not useful with the vanilla cgloss4e.sty, which ends
% glossed lines prematurely.
% I do have a suitably modified file, cgloss.sty. With it you can do the
% following:
% \gll To kimeno. \\
%      the text \\ \jambox{(Greek)}
% \trans `The text.'


\newdimen\jamwidth \jamwidth=2in
\def\jambox{\@ifnextchar[{\@jambox}
	       {\@ifnextchar*{\@jamsetbox}{\@jambox[\the\jamwidth]}}}

% Quickie invocation: The argument is delimited by the parentheses (no width
% argument allowed). I redefine it in my documents to add formatting.
% Syntax: \jam(Some note)
%
\def\jam(#1){\jambox{(#1)}}

% Set width AND display the argument.
% The star is read and ignored; the argument #1 is boxed, used to set
% \jamwidth, then passed to \@jambox (which also puts it in \@tempboxa!)
%
\def\@jamsetbox*#1{\setbox\@tempboxa\hbox{#1}\jamwidth=\wd\@tempboxa
  \@jambox[\the\jamwidth]{\box\@tempboxa}}

%% Version 1: old & stupid
%% \def\@jambox[#1]#2{\hfill\hbox to #1 {#2\hfil}}

% Version 2:
% Always takes up \jamwidth space, even if it means breaking the line. But it
% works on ragged-right mode, too.
% \def\@jambox[#1]#2{\setbox\@tempboxa\hbox {#2\hfil}%
%	\ifdim \wd\@tempboxa<#1\relax \wd\@tempboxa=#1\relax\fi
%	\hskip 0.5em plus 1fill
%	\penalty 100\vadjust{}\nobreak\hfill\box\@tempboxa\par}
% The penalty enables a break.  \vadjust inserts an empty element
% at the beginning of the next line, protecting \hfill from being discarded.

% Version 3:
% This seems to cover everything!  But unfortunately, it won't work in
% ragged-right mode-- the line is broken BEFORE the last word, to make enough
% space...
\def\@jambox[#1]#2{{\setbox\@tempboxa\hbox {#2}%
  \ifdim \wd\@tempboxa<#1\relax % if label fits in the alloted space:
    \@tempdima=#1\relax \advance\@tempdima by-\wd\@tempboxa % remaining \hspace
    \unskip\nobreak\hfill\penalty250 % break line here if necessary
    \hskip 1.2em minus 1.2em 	  % used when the line extends past the margin
    \hbox{}\nobreak\hfill\box\@tempboxa\nobreak
    \hskip\@tempdima minus \@tempdima\hbox{}%
  \else  % the label is too wide: just right-align it
    \hfill\penalty50\hbox{}\nobreak\hfill\box\@tempboxa
  \fi
  % suppress closing glue:
  \parfillskip=0pt \finalhyphendemerits=0 \par}}
% The penalty enables a break, taken only if the line cannot fit.
% The \hbox{} ensures the next line does not begin with \hfill, which would
% be discarded if initial.
% (\vadjust inserts an empty element at the beginning of the next line, so
% that COULD be used instead of \hbox{}).
% Algorithm adapted from The TeXBook.
%
% The closing \par could be a problem if there is a \parskip...

% \subsection{XPs}
% \input{gb4e-tests/bars}
% \input{gb4e-tests/crossrefref}
% \input{gb4e-tests/eal}  
% 
% \subsection{Styles for source line}
% \input{gb4e-tests/nontypo-ex}
% \input{gb4e-tests/typo-ex}
% The underlinings in the following examples should all be of the same height. The words should also be of the same height.

\ea
\gll This sentence has \ulp{underlined}{8} \ule{passages} and \ule{underlined} words\\
 normal normal normal short                     normal normal veryveryveryveryverylong end\\
\z

\ea
\gll \ulp{xxx}{4} \ulp{fff}{4} \ulp{jjj}{4} \ulp{jxf}{4} \ule{xxx}  \ule{jjj}  \ule{fjx}  \ule{fff} \ulp{underlined}{3} \ule{fassjaes} abd\\
a b c d e  f g h i j k\\
\z
% 
% \section{AVM}
% \begin{avm}
\@0\[\asort{eating}
actor & \@1 \\ 
theme & \@2 \]
\end{avm}
% 
% \section{Maths}
% 
$\frac{n!}{k!(n-k)!} = \binom{n}{k}$

% $\lim_{x \to \infty} \exp(-x) = 0$

\medskip

$\forall x \in X, \quad \exists y \leq \epsilon$

\begin{equation}
a^2 + b^2 = c^2
\end{equation}
% 
% \section{Trees}
% \Forest{
  [S [NP] 
    [VP [V  [\textit{eats}] ]
      [NP] ]]
}

% 
% \section{Fonts}
The following characters should display nicely

\begin{tabularx}{\textwidth}{Xl l >{\itshape}l>{\bfseries}l>{\scshape}l>{\ttfamily}l>{\sffamily}l}
\lsptoprule
 & code & rm & it & bf & sc & tt & sf \\
\midrule
Schrock\\
\midrule
 Combining Macron-acute & 1DC4 & a᷄ & a᷄ & a᷄ & a᷄ & a᷄ &  a᷄  \\
modifier letter small a & 1D43 & ᵃ & ᵃ & ᵃ & ᵃ & ᵃ & ᵃ \\
modifier letter small e & 1D49 & ᵉ & ᵉ & ᵉ & ᵉ & ᵉ & ᵉ \\
modifier letter small open e& 1D4B & ᵋ & ᵋ & ᵋ & ᵋ & ᵋ & ᵋ \\
modifier letter small capital i& 1DA6 & ᶦ & ᶦ & ᶦ & ᶦ & ᶦ & ᶦ \\
modifier letter small o & 1D52 & ᵒ & ᵒ & ᵒ & ᵒ & ᵒ & ᵒ \\
modifier letter small u & 1D58 & ᵘ & ᵘ & ᵘ & ᵘ & ᵘ & ᵘ \\
\\
Brindle\\
\midrule
modifier letter raised down arrow  & A71C & ꜜ & ꜜ & ꜜ & ꜜ & ꜜ & ꜜ \\
\\
Gabelentz\\
\midrule
latin capital letter egyptological alef & A722 & Ꜣ & Ꜣ & Ꜣ & Ꜣ & Ꜣ & Ꜣ \\
latin small letter egyptological alef & A723 & ꜣ & ꜣ & ꜣ & ꜣ & ꜣ & ꜣ \\
\\
Sasaki, TMNLP TC 3 i\\
\midrule
 & 8B1B & 講 & 講 & 講 & 講 & 講 & 講 \\
 & 8AC7 & 談 & 談 & 談 & 談 & 談 & 談 \\
 & 793E & 社 & 社 & 社 & 社 & 社 & 社 \\
\end{tabularx}



\input{font-tests/glyphs.tex} 
\subsection{Tie bars}
\subsubsection{Heights}
a͡o     
e͡i     
p͡f     
k͡p     
t͡͡ʃ    
H͡L     
x͡x     
X͡X     
x͡I     
I͡x     
I͡I     


\subsubsection{Widths}
xi͡ix    
xm͡mx    
xi͡mx    
xm͡ix   
\\
XI͡IX    
XM͡MX    
XI͡MX    
XM͡IX   
\\       
xI͡ix    
xM͡mx    
xI͡mx    
xM͡ix    
\\        
xi͡Ix    
xm͡Mx    
xi͡Mx    
xm͡Ix    


\subsubsection{With command \textbackslash hitie\{x\}\{y\}}

xx\hitie{i}{i}xx   
xx\hitie{I}{I}xx   
xx\hitie{i}{I}xx   
xx\hitie{I}{i}xx  
\\
xx\hitie{W}{W}xx  
xx\hitie{w}{W}xx  
xx\hitie{W}{w}xx  
\\                 
xx\hitie{I}{m}xx  
xx\hitie{m}{I}xx  
xx\hitie{M}{i}xx  
xx\hitie{i}{M}xx  
\\                  
xx\hitie{x}{M}xx  
xx\hitie{M}{X}xx  
\\                  
xx\hitie{\i}{W}xx  
xx\hitie{W}{\i}xx 
\\                  
xx\hitie{\i}{W}xx 
xx\hitie{W}{\i}xx \\

\subsubsection{With command \textbackslash hitier\{x\}\{y\}/\textbackslash hitiel\{x\}\{y\}}
xx\hitier{\i}{W}xx           
xx\hitiel{W}{\i}xx          
\\
with optional argument to adjust horizontal placement\\
xx\hitier[.9]{\i}{W}xx      
xx\hitiel[.9]{W}{\i}xx       
% 
% \section{Figures}\label{sec:tables} 
% 
% \section{Tables}\label{sec:tables}
% \input{table-tests/general}
% 
% \section{Crossrefs}
% \input{crossref-tests/general}
% 
% \section{Bibliography}
% \subsection{Plain}
 
\bigskip
\verb+\citet{Chomsky1957}+

        \citet{Chomsky1957}            


\bigskip
\verb+\citet{Chomsky1957,Comrie1981}+

       \citet{Chomsky1957,Comrie1981} 


\bigskip
\verb+\citep{Chomsky1957}+

       \citep{Chomsky1957}            


\bigskip
\verb+\citep{Chomsky1957,Comrie1981}+

      \citep{Chomsky1957,Comrie1981} 


\bigskip
\verb+\citealt{Chomsky1957}+

     \citealt{Chomsky1957}          


\bigskip
\verb+\citealt{Chomsky1957,Comrie1981}+

    \citealt{Chomsky1957,Comrie1981}


\bigskip
\verb+\citeauthor{Chomsky1957}+

   \citeauthor{Chomsky1957}         

  


\bigskip
\verb+\citeyear{Chomsky1957}+

   \citeyear{Chomsky1957}          

  

\bigskip
\verb+\citegen{Chomsky1957}+

    \citegen{Chomsky1957}          

   

 \subsection{Pages}
 
\bigskip
\verb+\citet[12]{Chomsky1957}+

        \citet[12]{Chomsky1957}            

 


\bigskip
\verb+\citep[12]{Chomsky1957}+

       \citep[12]{Chomsky1957}            

 


\bigskip
\verb+\citealt[12]{Chomsky1957}+

     \citealt[12]{Chomsky1957}          

 
 
  


\bigskip
\verb+\citeyear[12]{Chomsky1957}+

   \citeyear[12]{Chomsky1957}          

  

\bigskip
\verb+\citegen[12]{Chomsky1957}+

    \citegen[12]{Chomsky1957}          

   

 \subsection{Two arguments}
  


\bigskip
\verb+\citep[see][12]{Chomsky1957}+

       \citep[see][12]{Chomsky1957}            

 


\bigskip
\verb+\citealt[see][12]{Chomsky1957}+

     \citealt[see][12]{Chomsky1957}          

 
 
   
   

 
 
\subsection{Same author}
\bigskip
\verb+\citet{Chomsky1957}+

        \citet{Chomsky1957}            


\bigskip
\verb+\citet{Chomsky1957,Chomsky1965aspects}+

       \citet{Chomsky1957,Chomsky1965aspects} 

 
 
\subsection{AAB}
\bigskip
\verb+\citet{Chomsky1957}+

        \citet{Chomsky1957}            


\bigskip
\verb+\citet{Chomsky1957,Chomsky1965aspects,Comrie1981}+

       \citet{Chomsky1957,Chomsky1965aspects,Comrie1981} 

 
 
\subsection{ABA}
\bigskip
\verb+\citet{Chomsky1957}+

        \citet{Chomsky1957}            


\bigskip
\verb+\citet{Chomsky1957,Comrie1981,Chomsky1965aspects}+

       \citet{Chomsky1957,Comrie1981,Chomsky1965aspects} 

 
 
\subsection{AAA} 

\bigskip
\verb+\citet{Chomsky1957,Chomsky1965aspects,Chomsky1965cartesian}+

       \citet{Chomsky1957,Chomsky1965aspects,Chomsky1965cartesian} 
 
\subsection{AABA} 
\bigskip
\verb+\citet{Chomsky1957,Chomsky1965aspects,Comrie1981,Chomsky1965cartesian}+

       \citet{Chomsky1957,Chomsky1965aspects,Comrie1981,Chomsky1965cartesian} 
\bigskip
\verb+\citealt{Chomsky1957,Chomsky1965aspects,Comrie1981,Chomsky1965cartesian}+

       \citealt{Chomsky1957,Chomsky1965aspects,Comrie1981,Chomsky1965cartesian} 

 
\section{Ordering of names with diacritics} 
\citet{Circov1900,MeierCircovac1900}

\section{Cite work in same volume}

\citetv{Chomsky1957}

\citepv{Chomsky1957}


\section{Different entry types}
\subsection{incollection}
\citet{Meier2000}

\section{Bib fields}

The URL should be available in book entries like \citet{Url2001}.

There should be no warnings issued for entries with a complex date like \citet{ComplexYear2000}.
% \input{bib-tests/index}  
% \printbibliography[heading=references] 
% 
% \section{Index}
% This indexes \ili{German}

This indexes \isi{Noun phrase}\footnote{The \isi{verb phrase} is in the footnote.}

\newpage
\il{Deutsch| see{German}}
\is{NP| see{Noun phrase}}

\ilsa{German}{Germanic}
\issa{Noun phrase}{Nouns}

This indexes \name{Boris}{Johnson}, 
\newpage
who is the Prime Minister\ia{Prime Minister@Prime Minister}.

\newpage

This outputs
\name[Johnson, Boris]{Alexander Boris}{de Pfeffel Johnson} but indexes the same as above. 

\newpage

\ia{BoJo@BoJo| see{Johnson, Boris}}
\iasa{Prime Minister}{Johnson, Boris}
 
% 
% \section{Diagrams}
% \section{Plots}

\begin{figure} 
  \barplot{Person}{\%}{P01,P02,P03}{
	      (P01,19.47733441) 
	      (P02,04.99311069) 
	      (P03,01.22486586)
  }
  \caption{Ratio of fixation time in the caption area in relation to fixation time to the whole screen}
  \label{fig:barplot}
\end{figure}
 
% 
% \section{Floats}
%  

\lipsum[1]

\begin{table}
 \begin{tabular}{lll}
\lsptoprule
  a & b c & d\\
  1 2 & 3 & 4 \\
\lspbottomrule
 \end{tabular}
\caption{This is a short caption.}
\end{table}

\lipsum[2]

\begin{table}
 \begin{tabular}{lll}
\lsptoprule
  a & b c & d\\
  4 5 & 6 & 7 \\
\lspbottomrule
 \end{tabular}
\caption{This is a very long caption which stretches over several lines. It includes additional explanations of the table and helps the reader to interpret the content. It could be shorter, but here, it is really important that it is long.}
\end{table}


\lipsum[7]

\begin{table}
 \begin{tabular}{lllrllrlrll}
\lsptoprule
  a & b c & d& b c & d& b c & d& b c & d\\
  4 5 & 6 & 7 & 6 & 7 & 6 & 7 & 6 & 7 \\
  a & b c & d& b c & d& b c & d& b c & d\\
  4 5 & 6 & 7 & 6 & 7 & 6 & 7 & 6 & 7 \\
  a & b c & d& b c & d& b c & d& b c & d\\
  4 5 & 6 & 7 & 6 & 7 & 6 & 7 & 6 & 7 \\
  a & b c & d& b c & d& b c & d& b c & d\\
  4 5 & 6 & 7 & 6 & 7 & 6 & 7 & 6 & 7 \\
\lspbottomrule
 \end{tabular}
\caption{This is a very long caption which stretches over several lines. It includes additional explanations of the table and helps the reader to interpret the content. It could be shorter, but here, it is really important that it is long.}
\end{table}


\lipsum[8] 

\begin{table}
 \begin{tabularx}{\textwidth}{llXXXXXXXXXXXll}
\lsptoprule
  a & b c &   a & b c &   a & b c & d& b c & d& b c & d& b c & d\\
  4 5 & 6 &   4 5 & 6 &   4 5 & 6 & 7 & 6 & 7 & 6 & 7 & 6 & 7 \\
  a & b c &   a & b c &   a & b c & d& b c & d& b c & d& b c & d\\
  4 5 & 6 &   4 5 & 6 &   4 5 & 6 & 7 & 6 & 7 & 6 & 7 & 6 & 7 \\
  a & b c &   a & b c &   a & b c & d& b c & d& b c & d& b c & d\\
  4 5 & 6 &   4 5 & 6 &   4 5 & 6 & 7 & 6 & 7 & 6 & 7 & 6 & 7 \\
  a & b c &   a & b c &   a & b c & d& b c & d& b c & d& b c & d\\
  4 5 & 6 &   4 5 & 6 &   4 5 & 6 & 7 & 6 & 7 & 6 & 7 & 6 & 7 \\
  a & b c &   a & b c &   a & b c & d& b c & d& b c & d& b c & d\\
  4 5 & 6 &   4 5 & 6 &   4 5 & 6 & 7 & 6 & 7 & 6 & 7 & 6 & 7 \\
  a & b c &   a & b c &   a & b c & d& b c & d& b c & d& b c & d\\
  4 5 & 6 &   4 5 & 6 &   4 5 & 6 & 7 & 6 & 7 & 6 & 7 & 6 & 7 \\
  a & b c &   a & b c &   a & b c & d& b c & d& b c & d& b c & d\\
  4 5 & 6 &   4 5 & 6 &   4 5 & 6 & 7 & 6 & 7 & 6 & 7 & 6 & 7 \\
  a & b c &   a & b c &   a & b c & d& b c & d& b c & d& b c & d\\
  4 5 & 6 &   4 5 & 6 &   4 5 & 6 & 7 & 6 & 7 & 6 & 7 & 6 & 7 \\
\lspbottomrule
 \end{tabularx}
\caption{This is a very long caption which stretches over several lines. It includes additional explanations of the table and helps the reader to interpret the content. It could be shorter, but here, it is really important that it is long.}
\end{table}


\lipsum[3] 

\begin{figure}
\fbox{

\parbox{5cm}{abc\\

\hspace{3cm}def \\

\hspace{2cm}g~~hi

\vspace{3cm}y\\

\rule{.4\textwidth}{3pt} \\g}}

\caption{This is a short figure caption.}
\end{figure}

\lipsum[4-5]

\begin{figure}
 \fbox{\fbox{\parbox{5cm}{\Huge ~~~~def\\ \Large X \fbox{D} EF}}}
\caption{This is a very long caption dwelling on several aspects of this figure. It is rather extensive, but helps the reader to better make sense of this involved diagram.}
\end{figure}

\lipsum[5]



\begin{figure}
\fbox{

\parbox{\textwidth}{abc\\

\hspace{4cm}afasd \\

\hspace{1cm}agagdsg~~hi

\vspace{2cm}y\\

\rule{.4\textwidth}{3pt} \\g

\rule{.2\textwidth}{3pt} \\f

}
}

\caption{This is a very long caption dwelling on several aspects of this figure. It is rather extensive, but helps the reader to better make sense of this involved diagram.}
\end{figure}

\lipsum[4-5]

\begin{figure}
 \fbox{\fbox{\parbox{\textwidth}{\Huge ~~~~dref\\ \Large X \fbox{D EF}}}}
\caption{This is a short figure caption.}
\end{figure}

\lipsum[5]


 
%  
% % 
% \input{intonation-tests/tobi} 
%  
% \section{Langsci-lgr tests}
% \ABL
\ABS
\ACC
\ADJ
\ADV
\AGR
\ALL
\ANTIP 
\APPL  
\ART
\AUX
\BEN
\CAUS
\CLF
\COM
\COMP
\COMPL 
\COND
\COP
\CVB
\DAT
\DECL
\DEM
\DEF
\DET
\DIST
\DISTR 
\DU
\DUR
\ERG
\EXCL
\F
\FOC
\FUT
\GEN
\IMP
\INCL
\IND
\INDF
\INS
\INTR
\IPFV
\IRR
\LOC
% \M %
\N
\NEG
\NMLZ
\NOM
\OBJ
\OBL
% \P %
\PASS
\PFV
\PL
\POSS
\PRED
\PRF
\PRS
\PROG
\PROH
\PROX
\PST
\PTCP
\PURP
\Q 
\QUOT
\RECP
\REFL
\REL
\RES
% \S  %
\SBJ
\SBJV
\SG
\TOP
\TR
\VOC                    

\ea
\gll ceci n' est pas une pomme\\
     \DET{} \NEG{} \COP{} \NEG{} \INDF{} apple\\
\glt `this is not an apple'     
\z 
% 
% \section{Langsci-basic tests}
% \input{package-tests/basic} 
% 
% \section{Langsci-optional tests}
% \input{package-tests/optional} 
% 
% \section{Textbooks tests}
% \subsection{Simple boxes}

\lipsum[1]

\tblsbwboxlight{This box is light gray}{
\lipsum[2]
}

\lipsum[14]

\tblsbwboxdark{This box is dark gray}{
\lipsum[3]
}

\subsection{Lined boxes}

\lipsum[15]

\tblsbwthinsandwich{This box has thin lines around it}{
\lipsum[4]
}

\lipsum[5]

\tblsbwthicksandwich{This box has thick lines around it}{
\lipsum[6-7]
}


\subsection{Floating boxes}
\lipsum[8-9]

\tblsbwboxlight{This box is inline}{
\lipsum[8]
}

\begin{figure}
\color{red} This dummy figure is used to check whether the floating box interferes with figure numbering
 \caption{Dummy figure for counting}
\label{fig:dummycountfiguretbls}
\end{figure}


\begin{figure}
\tblsbwboxlight{This box floats to the top}{
\lipsum[9]
}
\end{figure}

\lipsum[11-12]

\begin{figure}
\color{red}The caption of this figure is only 1 higher than \figref{fig:dummycountfiguretbls}
 \caption{Dummy figure for counting. The uncaptioned figure with the floating box is not counted.}
\label{fig:dummycountfiguretbls2}
\end{figure}


\subsection{Boxes with icons}
\lipsum[13-14]


\tblsbwboxlight[bulb]{This light box has a bulb icon}{
\lipsum[15]
}

\lipsum[16]
%
% \tblsbwboxdark[glass]{This dark box has a looking glass icon}{
% \lipsum[17]
% }
%
\subsection{Multipage boxes}


\lipsum[18]

\tblsbwthinsandwich{This box has thin lines around it. The lines repeat on page breaks}{
\lipsum[19-23]
}

\lipsum[24-25]

\tblsbwthicksandwich{This box has thick lines around it. The lines repeat on page breaks}{
\lipsum[26-30]
}

\lipsum[31]
 
% \lipsum[1]

\tblscolboxlight{This box is light gray}{
\lipsum[2]
}
\renewcommand{\tblsfillcolour}{black!12}
\begin{mdframed}[style=tblsfilledbox,frametitle={This box is light grey}]
	\lipsum[15]
\end{mdframed}

\lipsum[3]

\tblscolboxdark{This box is dark gray}{
\lipsum[4] 
}
\renewcommand{\tblsfillcolour}{black!20}
\begin{mdframed}[style=tblsfilledbox,frametitle={This box is dark grey}]
	\lipsum[15]
\end{mdframed}
 
\tblscolframebox{This box has a yellow color frame}{
 \lipsum[5]
}
\setlength{\trennlinie}{.8mm}
\renewcommand{\tblsboxcolor}{lsYellow}
\begin{mdframed}[style=tblsframedbox,frametitle={This box has a yellow color frame with a book icon}]
	\lipsum[7-10]
\end{mdframed}

\lipsum[6]

 
\tblscolframebox[book]{This box has a yellow color frame with a book icon}{
 \lipsum[7]
}

\lipsum[8]


\tblscolthinsandwich{This box has thin lines above and below}{
\lipsum[6]
}
\setlength{\trennlinie}{.8mm}
\renewcommand{\tblsboxcolor}{black}
\begin{mdframed}[style=tblsbox,frametitle={This box has thin lines above and below}]
	\lipsum[14]
\end{mdframed}

\lipsum[7]
 
\tblscolthicksandwich{This box has thick lines above and below}{
\lipsum[8]
}
\setlength{\trennlinie}{1.5mm}
\renewcommand{\tblsboxcolor}{lsYellow}
\begin{mdframed}[style=tblsbox,frametitle={This box has thick lines above and below}]
	\lipsum[14]
\end{mdframed}

\lipsum[9]

\begin{mdframed}[style=yellowexercise,frametitle={This is the subtitle}]
	\lipsum[10-14]
\end{mdframed}
\lipsum[11]
\begin{mdframed}[style=greyexercise,frametitle={This is the subtitle}]
\lipsum
\end{mdframed}
\lipsum[13]
\begin{mdframed}[style=tblsbox,frametitle={Test}]
\lipsum[14]
\end{mdframed}
\begin{mdframed}[style=tblsfilledbox,frametitle={Test2}]
	\lipsum[15]
\end{mdframed} 
% \lipsum[1]

\tblssy{book}{Literaturhinweise (mit \texttt{\textbackslash tblssy})}{\lipsum[1]}

\tblssy[black!20]{bulb}{Übungsaufgaben (mit \texttt{\textbackslash tblssy})}{\lipsum[42]}

\tblssy[black!12]{law}{Definition (mit \texttt{\textbackslash tblssy})}{\lipsum[2]}

\tblssy{glass}{Ideen für die eigene Forschung (mit \texttt{\textbackslash tblssy})}{\lipsum[2]}

\tblsli[green]{.8}{Test (mit \texttt{\textbackslash tblsli})}{\lipsum[22]}

\tblsfi[black!20]{Another test (mit \texttt{\textbackslash tblsfi})}{Another text}

\tblsfr[lsYellow]{book}{Literaturhinweise (mit \texttt{\textbackslash tblsfr})}{\lipsum[5]}

\tblsfr{law}{Benutze die \texttt{lsSeriesColor} (mit \texttt{\textbackslash tblsfr})}{\lipsum[5]}

\tblsfd{lsLightGreen}{1.8}{Ein gerahmter Kasten mit \texttt{\textbackslash tblsfd} }{\lipsum[42]}

 
% 
% \section{Syntactic sugar}
% %%%%%%%%%%%%%%%%%%%%%%%%%%%%%%%%%%%%%%%%%%%%%%%%%%%%%%%%%%%%%%%%%%%%%
%%      File: langsci-optional.sty
%%    Author: Language Science Press (http://langsci-press.org)
%%      Date: 2016-01-16 16:47:43 UTC 
%%   Purpose: This file contains useful, but not essential, 
%%            macros for books using langscibook.cls
%%  Language: LaTeX
%%   Licence: 
%%%%%%%%%%%%%%%%%%%%%%%%%%%%%%%%%%%%%%%%%%%%%%%%%%%%%%%%%%%%%%%%%%%%%



% Heiko Oberdiek
% http://tex.stackexchange.com/questions/136644/vertical-space-in-interaction-with-figure-center-environment
\newcommand{\oneline}[1]{%
  \begingroup
    \sbox0{\ignorespaces#1\unskip}%
    \leavevmode
    \ifdim\wd0>\linewidth
      \hbox to\linewidth{%
        \hss\resizebox{\linewidth}{!}{\copy0 }\hss
      }%
    \else
      \copy0 %
    \fi
  \endgroup
}

\newcommand{\centerfit}[1]{%
  \begingroup
    \sbox0{\ignorespaces#1\unskip}%
    \leavevmode
    \ifdim\wd0>\linewidth
      \hbox to\linewidth{%
        \hss\resizebox{\linewidth}{!}{\copy0 }\hss
      }%
    \else
      \centerline{\copy0 }%
    \fi
  \endgroup
}

% Helps to fit verbatim onto one line:
% http://tex.stackexchange.com/questions/140593/shrinking-verbatim-text/
\usepackage{fancyvrb}
\newenvironment{fitverb}
 {\SaveVerbatim{rlwv}}
 {\endSaveVerbatim
  \sbox0{\BUseVerbatim{rlwv}}
  \begingroup\center % don't add indentation
  \ifdim\wd0>\linewidth
    \resizebox{\linewidth}{!}{\copy0}%
  \else
    \copy0
  \fi
  \endcenter\endgroup}

\VerbatimFootnotes


% http://tex.stackexchange.com/questions/73464/inserting-rtl-text-in-verbatim-environment?rq=1
% verbatim with RTL text

%\DefineVerbatimEnvironment{rtlverbatim}{Verbatim}{commandchars=+\[\]}


\newcommand{\ispackage}[1]{\if@noftnote%
\is{package!{\scshape #1}}%
\else%
\is{package!{\scshape #1}|fn{*}}%
\fi%
}

% breaks the index
%\usepackage{doc}


\newcommand{\ispackageb}[1]{
\is{package!\texttt{#1}|(}
}
\newcommand{\ispackagee}[1]{
\is{package!\texttt{#1}|)}
}

\newcommand{\isoption}[1]{\if@noftnote%
\is{option!\texttt{#1}}%
\else%
\is{option!\texttt{#1}|fn{*}}%
\fi%
}

\newcommand{\iscommand}[1]{\if@noftnote%
\index{#1@{\ttfamily $\backslash$#1}}%
\else%
\is{{#1@\ttfamily $\backslash$#1}|fn{*}}%
\fi%
}


\newcommand\displaycmd[2]{%
  \DescribeMacro{#2}\centerline{\cmd{#1}}}
  
% \mex considered evil, as we need explicit reference to examples in XML
% % The following allows for a quick reference to following or preceeding examples (\mex{1}) or
% % (\mex{0}) but also (\mex{-1})
% % taken from covington.sty (check)
% \newcounter{lsptempcnt}
% 
% \newcommand{\mex}[1]{\setcounter{lsptempcnt}{\value{equation}}%
% \addtocounter{lsptempcnt}{#1}%
% \arabic{lsptempcnt}}%

\newcommand{\fitpagewidth}[1]{
  \resizebox{\textwidth}{!}{#1}
}



%add intonation bars over morphemes or words
\newcommand{\intline}[2]{\settowidth{\LSPTmp}{#2}\raisebox{#1pt}{\parbox{.1mm}{\rule{\LSPTmp}{.5pt}}}#2}

%add rising or falling intonation
\newcommand{\dline}[3]{%
  \parbox{.1mm}{\begin{picture}(0,0)%
        \put(0,#1){\line(#2,-1){#3}}%
        \end{picture}%
  }%
}

%% rotated table headers
% create lengths
\newlength{\rotheight}
\newlength{\rotwidth}

\newcommand{\rotatehead}[2][1cm]{
%width is the width of the parbox
%height is the buffer space used to vertically stretch the headere
\setlength{\rotwidth}{#1}
\setlength{\rotheight}{.85\rotwidth}
  \begin{rotate}{33}~ %nbsp shifts the content away from the line underneath
   \parbox{\rotwidth}{\raggedright #2}   
  \end{rotate}%
  \rule{0pt}{\rotheight} %add zero width rule to get the right table height
}

% example metadata

\newcommand{\langinfo}[3]{{\upshape #1\il{#1}~(%
\ifx\\#2\\%
\else%
#2;
\fi%
#3)}\nopagebreak[4]\ignorespaces}

 
\newcommand{\fittable}[1]{\resizebox{\textwidth}{!}{#1}}


% integrate see also in multiple indexes
\def\igobble#1 {} 
\newcommand{\langsciseealso}{\par\addvspace{.1\baselineskip}\hspace*{1.4cm}\hangindent=1.4cm\seealso}
\newcommand{\ilsa}[2]{\il{#1@\igobble | langsciseealso{#2}}}
\newcommand{\issa}[2]{\is{#1@\igobble | langsciseealso{#2}}}
\newcommand{\iasa}[2]{\ia{#1@\igobble | langsciseealso{#2}}}

\usepackage{tabularx}
\usepackage{array}
\newenvironment{widetabular}[1][1]
  {\tabularx{#1\textwidth}}
  {\endtabularx}


\newcommand{\citetv}[1]{\citeauthor{#1} (\citeyear*{#1} [this volume])}
\newcommand{\citepv}[1]{(\citealt{#1} [this volume])}


% Vowel chart tikz commands
\newcommand{\aeiou}{%
	\node at (1.5,0) (a) {a};
	\node at (0,3) (i) {i};
	\node at (3,3) (u) {u};
	\node at (0.5,1.5) (e) {e};
	\node at (2.5,1.5) (o) {o};
}
\newcommand{\aeiouEO}{%
	\node at (1.5,0) (a) {a};
	\node at (0,3) (i) {i};
	\node at (3,3) (u) {u};
	\node at (0.25,2) (e) {e};
	\node at (2.75,2) (o) {o};
	\node at (0.75,1) (E) {ε};
	\node at (2.25,1) (O) {ɔ};	
}

%no hyphenation left alingned
\newcolumntype{Q}{>{\raggedright\arraybackslash}X}
%no hyphenation right aligned
\newcolumntype{S}{>{\raggedleft\arraybackslash}X}
%no hyphenation centered
\newcolumntype{C}{>{\centering\arraybackslash}X}
%no hyphenation fixed width
\newcolumntype{L}[1]{>{\raggedright\let\newline\\\arraybackslash\hspace{0pt}}m{#1}}
%no hyphenation centered fixed width
\newcolumntype{Z}[1]{>{\centering\let\newline\\\arraybackslash\hspace{0pt}}m{#1}}
%no hyphenation right aligned fixed width
\newcolumntype{R}[1]{>{\raggedleft\let\newline\\\arraybackslash\hspace{0pt}}m{#1}}

\newcolumntype{d}[1]{D{.}{.}{#1}} 
\newcommand{\xxref}[2]{(\ref{#1}--\ref{#2})}


% Underlining in gb4e-example Environments. Usual underlining commands that span multiple words do not work, because gb4e would parse it as one word. 
% Example: \underline{My example phrase} should become \ulp{My}{~~~~~} \ulp{example}{~~~~~} \ule{phrase}
% Note: The 2nd Argument of the \ulp command is filled in by experience - if you are not familiar with the command, you should experiment a bit. Usually, five tildes are enough, but be sure to check the outcome. 
% \ule is meant to be the last word in a phrase that is underlined. Therefore, \ule does not have an extra length.
\usepackage[normalem]{ulem} 
\usepackage{calc}
\newlength{\fulllength}
\newcommand{\ulp}[2]{%#1: stuff to underline, #2: extra length to skip the whitespace between to components
  \settowidth{\LSPTmp}{#1}%
  % several boxes are need to assure that words with ascending and descending letters are underlined at the same 
  % level, leading to the impression of a continuous stroke
  \parbox[t]{\LSPTmp}{ %restrict first box to the length of first argument
      \settowidth{\fulllength}{\parbox{\LSPTmp}{~}\parbox{#2mm}{~}} %inner box is larger than outerbox, so underlining will extend beyond length of outer box
      %             align parbox to bottom
      %              |           mbox to prevent hyphenation
      \uline{\parbox[b]{\fulllength}{\mbox{#1}}}
    }
}

\newcommand{\ule}[1]{%#1: stuff to underline, no extra length
  \ulp{#1}{0}
  }
  
  


\newcommand{\longrule}{\rule{1em}{.3pt}}
\usepackage{colortbl} 
\newcommand{\shadecell}{\cellcolor{black!20!white}}

% vertical alignment of  numbered  example
\newcommand{\eabox}[2][-.7\baselineskip]{
  \ea
    \parbox[t]{.8\textwidth}{
      \vspace{#1}
      #2
     } 
  \z
}
\newcommand{\exbox}[2][-.7\baselineskip]{
  \ex
    \parbox[t]{.8\textwidth}{
      \vspace{#1}
      #2
     }  
}

%fix \verb error in biblatex
\makeatletter
\def\blx@maxline{77}
\makeatother


\newcommand{\twodigitexamples}{\exewidth{(23)}}
\newcommand{\threedigitexamples}{\exewidth{(234)}}
\newcommand{\fourdigitexamples}{\exewidth{(2345)}}

\newcommand{\noabstract}{\vspace*{-2\baselineskip}} %for chapters without abstract

\newcommand{\rephrase}[2]{{\color{yellow!30!black}#2}\todo{replaced `#1'}}

\newcommand{\missref}[2][]{\todo[#1]{missing reference #2}}

\newenvironment{indentquote}[1]%
  {\list{}{\leftmargin=#1\rightmargin=0pt}\item[]}%
  {\endlist}
  
  
\newcommand{\phonrule}[3]{#1 $\to$ #2 / #3}
\newcommand{\featurebox}[1]{$\left[\begin{tabular}{>{\scshape}c}#1\end{tabular}\right]$}


\definecolor{RED}{cmyk}{0.05,1,0.8,0}

%connect two elements with lines
\newcommand{\connect}[2]{%
  \tikz[overlay,remember picture]{%
    \draw[-,thick] (#1) -- (#2) node   {};  %
  }
} 

\newcommand{\examplesroman}{
  \let\eachwordone=\upshape
  \exfont{\upshape}
}
\newcommand{\examplesitalics}{
  \let\eachwordone=\itshape
  \exfont{\itshape}
}

\newenvironment{modquote}[1][6mm]% slightly less indented quote for hyphenation issues
  {\list{}{\leftmargin=#1\rightmargin=0mm}\item[]}%
  {\endlist}
  
%%%%%%%%%%%%%%%%%%%%
%%%%           %%%%%
%%%%   PLOTS   %%%%%
%%%%           %%%%%
%%%%%%%%%%%%%%%%%%%%

% \newcommand{\barplot}[4]{%
%   \begin{tikzpicture}
%     \begin{axis}[
% 	xlabel={#1},  
% 	ylabel={#2}, 
% 	axis lines*=left, 
%         width  = \textwidth,
% 	height = .3\textheight,
%     	nodes near coords, 
% 	xtick=data,
% 	x tick label style={},  
% 	ymin=0,
% 	symbolic x coords={#3},
% 	]
% 	\addplot+[ybar,lsRichGreen!80!black,fill=lsRichGreen] plot coordinates {
% 	    #4
% 	}; 
%     \end{axis} 
%   \end{tikzpicture} 
% }
% \usepackage{pgfmath,pgfplotstable}
% \newcommand{\langsciplot}[2]{%% experimental
%         \pgfplotstablegetcolsof{#1.csv}
%         \pgfmathsetmacro{\langscicsvlength}{\pgfplotsretval-1}
%         \begin{tikzpicture}[trim axis right,trim axis left]
%             \begin{axis}[
%                     #2,
%                     xtick=data,
%                     axis lines*=left,
%                     nodes near coords,
%                     ymin=0,
%                     width=\textwidth]
%             \foreach \i in {0,...,\langscicsvlength} {
%                 \addplot[
%                     /pgf/number format/read comma as period
%                     ] table [x index={0},y index={\i}] {#1.csv};
%             }
%             \end{axis}    
%         \end{tikzpicture}
% } 
\backmatter
\phantomsection%this allows hyperlink in ToC to work
\printbibliography[heading=references]
\cleardoublepage

\phantomsection 
\addcontentsline{toc}{chapter}{Index} 
\addcontentsline{toc}{section}{\lsNameIndexTitle}
\ohead{Name \lsNameIndexTitle} 
\printindex 
\cleardoublepage
  
\phantomsection 
\addcontentsline{toc}{section}{\lsLanguageIndexTitle}
\ohead{\lsLanguageIndexTitle} 
\printindex[lan] 
\cleardoublepage
  
\phantomsection 
\addcontentsline{toc}{section}{\lsSubjectIndexTitle}
\ohead{\lsSubjectIndexTitle} 
\printindex[sbj]
\ohead{} 

\end{document}